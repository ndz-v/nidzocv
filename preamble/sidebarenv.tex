%----------------------------------------------------------------------------------------
%   Sidebar Environment
%----------------------------------------------------------------------------------------
\NewDocumentCommand{\@hRulefill}{}{
  {\color{accentColor}%
      \leaders\hrule height 0.5\@sBSectionLineWidth\hfill\kern\z@}
}

\NewDocumentCommand{\@ruleline}{m}{
  \par\noindent\raisebox{.6ex}{%
    \makebox[\linewidth]{\raisebox{-.6ex}{#1}\hspace{1ex}\@hRulefill}}%
}
% \NewDocumentCommand{\@ruleline}{m}{
% 	\par\noindent\raisebox{.6ex}{%
% 		\makebox[\linewidth]{\@hRulefill\hspace{1ex}%
% 			\raisebox{-.6ex}{#1}\hspace{1ex}\@hRulefill}}%
% }

\NewDocumentCommand{\displaySidebarSignature}{}{
  \vfill
  \centering
  \includegraphics[width=4cm]{\signature}\\
  \pgfkeysvalueof{city}, \today
}

\newif\if@isPictureDefined
\newif\if@isSidebarLeft
\global\@isPictureDefinedfalse

% Optional arguments: xshift, yshift, picture size
\NewDocumentCommand{\displaySidebarPicture}{O{0cm}O{0cm}O{0cm}}{
\IfFileExists{\profilePicture}{
  \global\@isPictureDefinedtrue
  \begin{tikzpicture}[remember picture, overlay]
    \node[@sidebarPicturePosition,
      yshift=\textheight+2\@margin,
      xshift=-3.5pt,
      minimum width=\@sideWidth+2\@margin-2pt,
      draw, line width = 2pt,color=backgroundColor,
      minimum height= 150pt,
      path picture={
          \node[xshift=#1,yshift=#2,anchor=center] at (path picture bounding box.center){\includegraphics[height=150pt+#3]{\profilePicture}};
        }
    ] at (sidebar.north east) (picture)
    {};

  \end{tikzpicture}
}{\global\@isPictureDefinedfalse}
\vspace{145pt}
}

\NewDocumentCommand{\displaySidebarName}{}{
  \begin{tikzpicture}[remember picture, overlay]
    \if@isPictureDefined
      \tikzset{@sidebarNameBlock/.style={}}
      \tikzset{@sidebarNameBlockAnchor/.style={anchor=north}}
      \coordinate (sidebarNamePos) at (picture.south);
    \else
      \if@isSidebarLeft
        \tikzset{@sidebarNameBlockAnchor/.style={anchor=east}}
      \else
        \tikzset{@sidebarNameBlockAnchor/.style={anchor=west}}
      \fi
      \tikzset{@sidebarNameBlock/.style={yshift=\textheight+\@margin}}
      \coordinate (sidebarNamePos) at (sidebar.base);
    \fi
    \node[@sidebarNameBlock,@sidebarNameBlockAnchor,rectangle,fill,blockColor,
      minimum width=\@sideWidth+2\@margin,
      minimum height=\@nameBlockHeight,
      align=center,
      text width=\@sideWidth,
      drop shadow] at (sidebarNamePos)
    {%
      \LARGE\color{white}
      \pgfkeysvalueof{firstName}\hspace{2\@idNameSep}\textbf{\pgfkeysvalueof{lastName}}\par};
  \end{tikzpicture}
  \vspace{\@positionSep}
}

\NewDocumentCommand{\displayJobDetails}{}{
  \begin{center}
    \pgfkeysvalueof{position}\\
    \pgfkeysvalueof{positionid}
  \end{center}
  \vspace{\@positionSep}
}

% Left Sidebar
\NewDocumentEnvironment{sidebar}{}{
  \global\@isSidebarLefttrue
  \tikzset{@sidebarPicturePosition/.style = {anchor=north east},}
  \tikzset{@sidebarNameBlockAnchor/.style={anchor=east}}
  \refstepcounter{cvPage}\label{cvPage.\thecvPage}
  \begin{tikzpicture}[remember picture,overlay]
    \fill[backgroundColor](current page.north west)rectangle%
    ++(\@sideWidth+2\@margin,-\paperheight) node [anchor=north] (sidebar){};
  \end{tikzpicture}
  \color{fontColor}
  \begin{flushleft}
    \begin{minipage}[c][\@minipageLength]{\@sideWidth+0.2cm}
      }{
    \end{minipage}
  \end{flushleft}
}

\NewDocumentEnvironment{sidebar*}{}{
  \global\@isSidebarLeftfalse
  \tikzset{@sidebarPicturePosition/.style = {anchor=north west},}
  \tikzset{@sidebarNameBlockAnchor/.style={anchor=west}}
  \refstepcounter{cvPage}\label{cvPage.\thecvPage}
  \begin{tikzpicture}[remember picture,overlay]
    \fill[backgroundColor](current page.north east)rectangle%
    ++(-\@sideWidth-2\@margin,-\paperheight) node [anchor=north] (sidebar){};
  \end{tikzpicture}
  \color{fontColor}
  \begin{flushright}
    \begin{minipage}[c][\@minipageLength]{\@sideWidth}
      }{
    \end{minipage}
  \end{flushright}
}

\NewDocumentEnvironment{profile}{O{\pgfkeysvalueof{/variables/profile}}}{%
  \vspace{\@sectionSBSep}
  \@ruleline{#1}
  \vspace{-\@sectionSBSep}\\
}{
  \vspace{.3cm}
}

\NewDocumentCommand{\@contactTemplate}{mm}{
  \NewDocumentCommand{#1}{m}{
    \if@contactfirst \\ \else %
      \global\@contactfirsttrue \fi\node[@contactIcon]{#2};
    \pgfmatrixnextcell \node[@contactText]{##1};
  }
}

\NewDocumentCommand{\@contactTemplateLink}{mm}{
  \NewDocumentCommand{#1}{mm}{
    \if@contactfirst \\ \else %
      \global\@contactfirsttrue \fi\node[@contactIcon]{#2};
    \pgfmatrixnextcell \node[@contactText]{%
      \IfEq{##1}{}{##2}{\href{##1}{##2}}
    };
  }
}

\NewDocumentCommand{\contactTemplate}{O{} mm}{
  \if@contactfirst%
    \\%
  \else %
    \global\@contactfirsttrue %
  \fi %
  \node[@contactIcon]{#2};
  \pgfmatrixnextcell \node[@contactText]{%
    \IfEq{##1}{}{##3}{\href{##1}{##3}}
  };
}

\newif\if@contactfirst
\NewDocumentEnvironment{contact}{O{\pgfkeysvalueof{/variables/contact}}}{%
  \vspace{\@sectionSBSep}
  \@ruleline{#1}
  \vspace{-4\@sectionSBSep}\\

  \global\@contactfirstfalse
  \@contactTemplate{\address}{\faMapMarker*}
  \@contactTemplate{\phone}{\faPhone}

  \@contactTemplateLink{\email}{\faEnvelope}
  \@contactTemplateLink{\website}{\faGlobe}
  \@contactTemplateLink{\github}{\faGithub}
  \@contactTemplateLink{\linkedin}{\faLinkedin}
  \@contactTemplateLink{\twitter}{\faTwitter}
  \@contactTemplateLink{\keybase}{\faKey}
  \begin{tikzpicture}[every node/.style={inner sep=0pt, outer sep=0pt}]
    \matrix [
      column 1/.style={anchor=west,align=right},
      column 2/.style={anchor=west,align=right},
      column sep=\@contactItemSep,
      row sep=1.5\@contactItemSep,
      inner sep=0pt,
      outer sep=0pt] (contact)
    \bgroup
    \address{\pgfkeysvalueof{address}}\\[-5pt]
    \node[@contactIcon]{};
    \pgfmatrixnextcell \node[@contactText]{%
      \pgfkeysvalueof{zipcode} \pgfkeysvalueof{city}
    };
    \email{mailto:\pgfkeysvalueof{email}}{\pgfkeysvalueof{email}}
    \phone{\pgfkeysvalueof{phoneNumber}}
    }{%
    \\\egroup;
  \end{tikzpicture}%
  \vspace{.3cm}
}

\newcounter{@languages}
\setcounter{@languages}{1}
\pgfkeys{
  /languages/.is family,
  /languages/.unknown/.style = {%
      \pgfkeyscurrentpath/\pgfkeyscurrentname/.initial = #1
    }
}

\newcounter{@languagelevel}
\NewDocumentEnvironment{languages}{}{%
  \NewDocumentCommand{\languageRating}{mm}{%
    {\globaldefs=1\relax\pgfkeys{%
          /languages/lang\the\value{@languages} = ##2}}
    \node[]{##1}; \pgfmatrixnextcell %
    \node[@progressArea] (@language \the\value{@languages}) {}; \\
    \stepcounter{@languages}
  }%
  \@ruleline{\pgfkeysvalueof{/variables/languages}}%
  \vspace{15pt}
  \begin{tikzpicture}[every node/.style={text depth=0pt,inner sep=0pt,outer sep=0pt}]
    \matrix [column 1/.style=@column1,column 2/.style=@column2,%
      column sep=3\@sectionSBSep,row sep=1.5\@sectionSBSep] (contact)
    \bgroup
    }{%
    \\\egroup;
    \setcounter{@languagelevel}{1}
    \loop\ifnum\value{@languagelevel}<\value{@languages}
    \draw (@language \the\value{@languagelevel}.west) %
    node[@progressBar,minimum width=\pgfkeysvalueof{/languages/lang\the\value{@languagelevel}}em]%
    {};
    \stepcounter{@languagelevel}
    \repeat
  \end{tikzpicture}
  \vspace{.3cm}
}
